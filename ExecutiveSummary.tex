\section*{Executive Summary}
\addcontentsline{toc}{section}{Executive Summary}


This report is intended to be used as a source of comprehensive feedback and recommendations for the future handling of the UCL MSc course in mechanical engineering. The issues with the course are significant, and range from problems with the quality of module resources, the attitude and behaviour of lecturers, almost all aspects of the coursework exercises, the difficulty and focus of content, the administration and organisation, and the response to feedback and trust in the department. The problems are described, evidenced, and some suggestions are made on how to respond to them.

The lecture content is delivered via slides, although these are treated as a substitute for lecture notes. This leads to slides that do not function well as presentation aids or as tools for independent revision. The quality of the lecture slides is also very often extremely poor, not even meeting the standard expected of a GCSE student. The Microsoft Office abilities of some lecturers is concerning to say the least, and training will be needed for them to be able to produce decent materials, for example in {\LaTeX}. The shocking state of some of their figures demonstrates a lack of care; lecturers need to be held to a higher standard and pass a quality review.

The coursework marking is extremely sloppy and markers need to be given more time and motivation to mark properly. The pass mark needs to be lowered to encourage markers to use a wider range, and they need to stop arbitrarily capping marks around the 85\% mark. Marking feedback is often wrong and almost always insufficient and unhelpful, markers need to put in much more effort into giving decent feedback, for example by annotating the pdf. The department also needs to look into how long each coursework reasonably takes and assign the weights appropriately as currently this is not the case. Exercises should also be made more open to give a higher ceiling and allow for greater individuality and creativity. Review processes are needed in the marking and design of coursework briefs.

The content of the course is lacking in ambition and is far too simple for a univeristy as high ranking as UCL. Many of the modules would not feel out of place as A level courses and the department should focus on making large changes to the specifications to make the course more suitable for postgraduates. Advanced mathematics is missing from the course and this should be a warning sign\footnote{Matrix multiplication does not count as advanced mathematics}. Examinations resources are badly organised, and the papers are often poorly constructed. The main issues are the distribution of marks which appear random at times, and the writing of questions where the author does not take the aims of the examination into account. I also have significant issues with the administration rarely listening to our problems.

The group design module was handled extremely poorly and serves as evidence of the lack of oversight into how the module coordinators run the modules. The project selection should be changed to where the students sign up to the projects to decide teams instead of the students deciding the teams and then getting projects based on a first come first served basis. The priorities of the academic integrity policy are confused, and it should be reviewed with a focus on actually malicious behaviour. The department could also reduce academic misconduct via choosing open coursework exercises that are more individual, and online examinations without easily searchable answers.