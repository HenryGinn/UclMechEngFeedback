\section{Course Overview and Administration}
\label{sec:CourseOverview}

The standard of difficulty in the engineering department is a long way off where it should be, and doesn't provide the expected level of academic challenge. The general organisation of almost all aspects of the course from the Moodle to examinations is a mess, and needs to be fixed. Perhaps most importantly, I also discuss the poor handling of feedback, and the lack of trust that we are able to have in the department.

\subsection{Course Difficulty}

This course is too easy for a university that ranks as high as UCL does. Here it is possible to waltz out with a first class degree, something that should not be the case for a high-calibre university. In my undergraduate course I felt intellectually capped at a high 2:1, and I actually preferred to be in that situation as attaining a first actually meant something. If UCL wants to be talked about in the same breath as Oxford or Cambridge then they need to seriously reevaluate the difficulty of the course. Below I outline some of the reasons why this is currently not the case, such as the lack of depth, problem solving, and advanced mathematics.

I want to first discuss UCL's confusion between mathematical ability and numerical competency. Being able to use some relationships, rearrange equations, and substitute numbers are skills tested in A-level physics, this is not what it means to do mathematics. The ``mathematics" at UCL could be done by anyone vaguely numerically literate who knows how to use a spreadsheet, and the wrong skills are tested. While there is a place for those skills in the course, the main focus in my opinion should be on the development of the mathematics used. Producing a mathematical model to describe situations requires a strong command of the underlying concepts and strong problem solving abilities, and such an engineer is far better equipped to handle new situations.

The problems always seem to be the same - apply some equations, plug numbers in, get answer. This is really boring and makes the examinations very limiting as the questions have likely been encountered before, just with different numbers. UCL could make things far more interesting if the questions had some spice to them and demanded actual problem solving skills. In my undergraduate course if you did not have an intimate understanding of the topics and could not come up with something original in the exam then you would do really badly, there were no free rides there. As an example of a more complicated question, a change to a model that was studied in the course could be proposed, and candidates would be asked to derive similar results to those found in the course when working with the new model.

I was surprised by the lack of more complicated mathematics in this course, there were barely even any differential equations or integrals. Given the huge preponderance of phenomena that are modelled by differential equations, this seems really odd. Linear systems are one of the only classes of problems where solve large problems can be solved and therefore many problems are coerced into being linear, so the almost complete lack of linear algebra was also unexpected. Probability and statistics are other areas that an engineer needs to be well-versed in, and these topics barely featured at all in the course. UCL would benefit from including more advanced mathematics into the course, especially as this also allows for more complicated material. Here I am talking about mathematics that goes beyond what is learned in A-level further mathematics.

I once mentioned the difficulty of the course using the computational fluid dynamics (CFD) coursework as an example, and I asked what was actually intellectually demanding about the exercise. They told me that lots of students had reported that it was too hard, and without trying to sound pretentious, those students just need to get good. UCL needs to stop catering to the lowest common denominator in this way, if people are not capable of working at the highest level then they should not be at a top university. The excuse that it needs to be accessible to those who have not done CFD before is weak as I am one of the people who had not done it before and it was still basic\footnote{This is not because of any intelligence on my part, I only got a 2:1 in my undergraduate course and am not exceptional.}. I am perplexed at how UCL can maintain a reputation of being a top university when offering such little challenge.

If UCL feels that they are limited by the calibre of their students then they should consider stricter entry requirements, or additional steps such as an entrance exam. Passing the admissions process at Oxford was a very reliable indicator that a student was capable of handling the difficulty of the course, these systems work. I would also like to note that many of my friends and I already do not feel as academically challenged as much as we should be. UCL already has students who are more capable, and is currently leaving us disappointed. From my experience I also suspect that many of the international students would have not have been accepted if they were not paying extortionate fees, this needs to be reviewed.

\subsection{Examinations}
% Structure and organisation
% Content and timings
% Open book exams
% How much is one mark worth?
% Examiners reports

The organisation of past papers in the engineering department is extremely chaotic. The questions on each paper are sometimes split up which further adds to the mess, and makes finding answers harder. It is evident that any attempt at organising them is either very weak or non-existent, and I have found no logic to the structure. Figure~\ref{fig:ClutteredPastPapers} shows all the examination resources provided for one of my modules as evidence of the lack of structure. In my undergraduate course all the past papers were stored in one place, organised by academic year, module, and finally year. Could UCL implement something similar please? The link to past papers on the Moodle just searches the library for the module code and did not return anything when I tried this.

\begin{figure}[H]
    \centering
    \includegraphics[width=0.4\textwidth]{Figures/ClutteredPastPapers.png}
    \caption{An example demonstrating the lack of structure in the organisation of past paper resources}
    \label{fig:ClutteredPastPapers}
\end{figure}

The structure of the exams themselves have issues. In my undergraduate course each exam had the same structure: 90 minutes per paper, three questions where candidates pick two, and 25 marks per question. This was almost universal across the department\footnote{In the first year there were slight deviations to this due to the structure of the modules.}, and this practice went back at least ten years. All papers were written in the same style with the same template, and again this was universal throughout the department. At UCL there is no such consistency between modules, or even for the same module between years. I had no idea what the structure of my examinations would be apart from a handful of small details. Please can UCL pick a format they like and stick to it, this is more than possible to do.

The number of errors in the past papers and solutions that we were provided was inexcusable. I can only hope that those were the first drafts and were not actually used as otherwise this would be a very serious error on UCL's part. Examples of errors were things such as giving a power when work was asked for, forgetting to multiply/divide by efficiencies, substituting in the wrong numbers, and errors in the questions that make it ambiguous or impossible. In general they were sloppily done and I refuse to believe that they passed a peer review process. UCL needs to address this immediately and ensure there is a robust system in place to lower the error rate to near zero in both question papers and mark schemes. The corrected versions should also be supplied as resources for following years. For comparison, in my undergraduate the only error was an incorrect number which was corrected before we started the paper.

The way the papers are written is often poor and shows bad understanding of how a test should be constructed. For example if someone has already been tested then it does not need to be tested again. There are two reasons for this, firstly because it is unnecessary, and secondly because someone who does not know that part gets penalised twice. There is more than enough content in the course to fill a paper, they do not need to cover the same knowledge or competencies twice. The prime example of this would be the power trains and auxiliary machinery examination where 42\% of the paper was dedicated towards using one method of duct sizing for an HVAC system. This is especially perplexing given that we learn two methods of doing this, and they could have very easily had one question for each method rather than two questions on one method.

The second reason why the papers are badly written is harder to fix, and will probably require training of staff. If a student has a partial understanding then they should still be able to make some attempt at the question. For example if a student was unable to do the first part of the question and the second part relies on the first part, they will lose the marks for both sections. This situation could be fixed by having the first part show that something is true. The question remains unchanged and the student still needs to do the work, although now they can attempt the second question using the given piece of information. If a very long computation is asked for as a single question then this suffers similar issues. In general the question authors should keep in mind how much they want to penalise student for each failure of understanding, and structure the questions to match.

Something that caused confusion during revision and in the examinations was the strange distribution of marks. The number of marks is meant to be indicative of how long an answer should be, in terms of word count or steps in a computation. This regularly did not hold however, leading us to believe we were either overcomplicating or oversimplifying answers. This is bad form for a question writer as each mark should take around the same amount of work. It is acceptable for marks to not be equal in difficulty however, for example in my undergraduate exams the difficulty increased throughout the question. The marks were still allocated to how many units of work were needed to answer each part.

UCL does not understand the online format for examinations. As students have unrestricted access to the internet and all their resources the papers need to account for this, but UCL goes about this in the wrong way. They did this by significantly increasing the time pressure, and making a very intense experience. These were not hard due to the content, they were hard to the lack of time, something that should not be the case for an examination. I had some online examinations in my undergraduate due to COVID-19, and they were made suitable by ensuring that the answers were not be easily accessible on the internet. The reason that ChatGPT is a threat to online examinations at UCL and not at Oxford is because the latter are actually hard, and ChatGPT does not have the problem solving abilities to do them.

One of the issues I faced with online examinations at UCL was with prepared materials. For one of our modules we are reliably asked to carry out something called the rainflow algorithm, and I had prepared a program to do this for me. I asked the module coordinator about the legality of this and got a politician's response, and overall they were very uncooperative. They told me that I needed to show all my working, although I had described the output of the program and how it showed all the steps of working that the lecturer did. When asking if they could tell me what steps of working I had missed, they were not helpful. They also said that they could not answer without revealing the content of the examination, although this does not hold water. I was asking a hypothetical question, ``If $P$, does $Q$ follow", they did not need to reveal the truth of $P$ in order to answer this.

Personally I do not think they anticipated that anyone would do this and I was throwing a spanner in the works. There would be nothing stopping me sharing my code with the whole cohort before the examination started. If they answered yes, then the question becomes completely redundant as all difficulty is removed. If they answered no, then there would still be nothing stopping me from copying the output of the program by hand. My answer would be indistinguishable from someone who did the whole question as intended, and the question is still redundant. I did eventually get a sufficiently straightforward answer and used it in the examination, but I think this is a good example of the lack of planning or foresight of UCL. The actual resolution to this problem would be to not include this algorithm in an online examination at all.

UCL does not appear to do examiners reports and personally I think they should. In my undergraduate course the would publish a document a few months after the results, and this would be available as part of the past paper archive. This included a description of the curving process, an anonymised ranking of results\footnote{This was the cumulative total of people who got more than each percentage value.}, and some statistics about gender breakdown of results, comparisons with previous years, etc. These documents were not just interesting to those who did the examinations, they were useful for future years as the markers gave their comments about the performance on each question.

\subsection{Administration and Feedback}
\label{sec:AdministrationAndFeedback}
% Lack of trust in the system
% Roberts building access
% Moodle organisation
% Some lecturers were strangely protective of their content, insisting on releasing it week by week
% Module problems

The Moodle pages are extremely badly organised and they were a source of constant frustration to me. Lecturers often split the resources up lecture by lecture, but this was really annoying and just meant that we had to scroll more. If we needed to find some information, we may have known what topic it was but it was hard to remember what week of lectures it was. Having all the lecture slides in one document eliminates these issues, and I would usually download them all and combine them myself anyway. While my undergraduate course was not always perfect in this regard, figure~\ref{fig:PageOrganisationOxford} shows what almost all module pages would look like. For comparison the Moodle pages at UCL would not fit in 10 figures, but figure~\ref{fig:PageOrganisationUCL} shows a snippet demonstrating the mess.

Scrolling through the new and renewable energy systems page requires pressing page down 21 times, yet in my undergraduate we often did not even need to scroll at all. I appreciate the supplementary material that the lecturers provide, although this either belongs in an appendix, a reading list, or in a section giving references to additional resources. The occasional lecturer were also strangely protective of their resources and insisted on releasing them week by week. A few lecturers also uploaded content in the announcements tab which is separate to where all the other content is. The reading list links on the side rarely work either, and the exam resources never work as previously mentioned. These are a minor issues, but makes using the Moodle awkward, and can be easily fixed.

The way the lecturers use the question and answer section on Moodle needs to improve. Firstly, the forum is designed to be a communal space, and I see no use case for the private response feature. Lecturers should stop using this completely, and personally I think the feature should be removed. The response rate from some lecturers was also poor\footnote{I want to give credit to Richard Bettany here. He asked us to contact him via email and he was very responsive, even after usual hours and on weekends. He went far above the standard that I would expect.}, with a worrying number of questions getting ignored\footnote{One exception I want to make to this is William Suen who did not respond to me on Moodle, but instead dedicated 90 minutes for a one-on-one meeting with me to discuss my questions. This is also going above and beyond what I expect from the lecturers.}. In general some lecturers were very hard to contact. I have a friend who took weeks trying to contact a lecturer, including daily emails and waiting by their office - this is not good enough.

\newpage
\vspace*{10mm}
\begin{figure}[H]
    \centering
    \includegraphics[width=\textwidth]{Figures/PageOrganisationUCL.png}
    \caption{The organisation of resources at UCL. This is for two hours of lectures.}
    \label{fig:PageOrganisationUCL}
\end{figure}

\newpage
\vspace*{10mm}
\begin{figure}[H]
    \centering
    \includegraphics[width=\textwidth]{Figures/PageOrganisationOxford.png}
    \caption{The organisation of resources for the Oxford undergraduate mathematics course. This is for a 16 hour lecture series.}
    \label{fig:PageOrganisationOxford}
\end{figure}
\newpage

Access to the Roberts building should be extended. Currently it opens at 7:30~am, and the basement opens half an hour later than that. This is not a staffing issue, security are in the building before this time and academics are allowed in. Another study room, 410, is also often locked at 7:30~am. I travel off-peak and arrived around 7:20~am, I would not be able to access the work rooms for an awkward 40 minutes. They could fix these issues by trusting the students and letting us in at the same time as academics, including when the basement opens and room 410 is unlocked. These problems are worse on the weekends and bank holidays.

I was surprised by how few modules were on offer at UCL. There were only five choices where we pick three, and the other five were compulsory, and as a comparison my undergraduate course had around 35 modules to choose from each year. I understand that this course is designed to be accessible to those who have not come from a mechanical engineering background, although even with this restriction the selection could have been wider. We were also not allowed to take additional modules which I thought was odd and needlessly restrictive. Again in comparison to my undergraduate, we were allowed to take as many extra modules as we wanted if we were capable, and I knew someone taking on a 50\% additional workload. At UCL we can still attend these lectures, but you would not be allowed to take the examinations. This policy seems unacademic and unnecessary.

While this only affected a few students, there were some who were not allowed to take advanced computer applications in engineering. These students were from another MSc degree offered by the engineering department and had lower priority over mechanical and power systems students. I think this is disgraceful and the department should have been flexible enough to allow them on the module, for example by adding an extra row of chairs at the front of the lecture hall\footnote{This would not have been an issue anyway. In the first lecture the large hall had standing room only, but a few weeks in there was only around 20 students who turned up as these lectures were extremely boring and useless}. In my undergraduate studies the modules would be run no matter how many people were doing them.

Feedback from the department is a key issue that needs to be a focus. Out of all my complaints, the fact that the department seems unwilling to change or listen in any meaningful way is the largest of them. They regularly deny the existence of our problems and seem completely disconnected from the student experience. I have no evidence for this beyond hearsay, but I have even heard of the department fabricating good feedback results for a form that we never received. I have little to no confidence in the department to enact positive changes, and I do not trust them. They have received significant amounts of feedback which they seemingly have not acted on.

Beyond listening to feedback, I do not trust the department in general. I feel that I cannot rely on basic things such as unambiguous question phrasing in exams or for coursework briefs to make sense. Permeating through everything there seems to be a sloppiness and lack of care that is not acceptable. Given that this course cost £18,000 it feels like a scam, and I cannot imagine how angry the international students must be. UCL is raising the price to £19,300 next year which I think is disgraceful. As a part of a prestigious university UCL engineering should be striving to be the best, but instead they have a ``that will do" attitude that makes it seem like a cash grab. They should start by implementing stringent peer review processes and holding themselves to significantly higher standards.