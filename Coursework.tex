\section{Coursework}
\label{sec:Coursework}

% Change earlier reference to feedback to the lecturer feedback in this section

The majority of my time on this course was spent doing coursework and was where a significant amount of problems with the course came from. The biggest issues are to do with the unreliable and unrealistic marking, although I also think the exercises are too closed and restrictive, and their weightings are badly chosen.

\subsection{Marking of Coursework}
\label{sec:MarkingOfCoursework}

One of the largest sources of frustration for everyone I know was the bad marking of coursework. The marking did not feel representative of the quality of work, there was a strong aversion to giving very high marks, and choices were often unreasonable. I have particular qualms with the marking of presentation quality. My opinions in this subsection will be based mainly on my own work, but also from where I proof-read the work of my friends and knew their marks.

I have several friends who started most of their coursework exercises a few days before the deadline, and the marks they attained were similar to my friends who spent weeks, only losing a handful of percentage points. I know that the work is marked based on its quality and not the time and effort put into it, but I proof-read some of their work and I can confirm that these rushed coursework submissions were significantly worse than the polished ones. While there was some correlation between effort and mark attained, this was far weaker than it should be given the good correlation between quality and effort.

I believe that part of this problem is due to the very narrow range of marks given. I can understand why a marker would be incentivised against giving out marks below 50\% (the pass mark), so half of the range is already lost. Lecturers do not like giving out high marks, so for all practical purposes they are marking in the range from 50\% to 85\%. This is not wide enough, it means that the difference between something done in a single night is only 35\% away from somebody's passion project. I would change the pass mark to 20\%, and make sure that the entire range from 0\% to 100\% was used.

Attaining extremely high marks seems to be impossible at UCL, and this is not due to the coursework exercises being difficult, it is entirely down to pointless battles against the markers. Most of my friends and I went above and beyond when it came to coursework. I regularly spent three to four times longer than I was meant to on coursework, exploring every detail. They required very little conceptual understanding or skills, and you would struggle to find any evidence of a lapse of understanding in the work I submitted. Why do I still lose a fifth of the marks available?

In the marking of presentation I would consistently lose around 20\% of the marks available for presentation which I think is ridiculous. My grammar is not perfect and I am not a professional typesetter, but the presentation quality of my work definitely meets and exceeds the standards that can be expected of a masters student. It frustrates me that the same people who use pictures so blurry you cannot even read them would deem my presentation to be lacking. Before the reader judges me based on the sloppiness of this document, I would like to note that this has been quickly slapped together so it is not up to my usual standard.

The lack of comprehensive and precise style guides is also an issue. I understand that some parts of presentation are uncontroversial (high resolution or vector figures, sensible font and margins, correctly cross-referenced equations and figures, etc.), but some lecturers have many personal preferences that they do not state. For example in my undergraduate we were told to write actively instead of passively, and this is the opposite of many UCL lecturers preference. From my experience lecturers seem to believe that they are far more comprehensive and precise than they actually are, and so they should start with a template guide. I recommend lecturers modify a version of the IEEE style guide to their liking~\cite{IEEEStyleGuide}.

Lecturers are also occasionally unclear in their rubrics, especially for presentation. Personally I believe presentation should cover the aesthetics of the document, the layout, and writing style, for example eliminating widows, orphans, and runts and using vector images. I have seen many other qualities being included in presentation however, such as the quality of the introduction, or even ``anything that is not mentioned elsewhere in the rubric". Anything to do with the content itself should not fall under presentation, this can be fixed by a review of the marking rubrics.

The rest of the marking has similar issues, and if I were to have logged every grievance I had then this document would be ridiculously long. Many people I know have already complained numerous times to lecturers about marking and only exceptionally rarely got anywhere, and the student reps have also held meetings about this. We are not unreasonable and we are not making these issues up, the department cannot deny the problems with the marking as they have significant evidence.

One of my friend's was told by a lecturer that they were removing a few marks because they were too close to 100\%. Lecturers do not want to give high marks, and once they find an excuse they will cut off a huge portion of the marks. We are simply up to the whips of the markers here, and I imagine even an expert would not get 100\% on these coursework exercises even if they dedicated their life to it. The lecturers appear to have the attitude that if we get above 70\% then we should be happy. Some of us are pouring our lives into these coursework and working 70+ hour weeks, so no, we are not happy with just 70\%.

I would like to see the module coordinators attempt each coursework exercise and I guarantee they would only be getting around 90\% or less. This is not a hypothetical argument, I think it would be a good test of whether the marking was reasonable, and should be implemented for future years. Perhaps this would even define the 100\% mark, and all marks would be scaled accordingly. If the department has a problem with this system then they are either accepting that their lecturers are not capable or that it is too hard to get 100\%. Their submission should also be published after results are released.

There are also far too many mistakes in the marking. I have seen people marked down for not including things that they included, the marker is objectively wrong in these instances. I have also seen two people with similar coursework submissions get wildly different marks, and someone who submitted code that did not even run get a first. I have also had situations where the module coordinator suggests something, and then get marked down when I followed their advice.

\subsection{Style and Nature of Exercises}

The coursework exercises we are given are often boring and unvaried. I believe they could be improved by allowing for longer form styles with less of a focus on being concise, and more of a focus on depth and narrative. The exercises are also not open enough and do not encourage originality and deep exploration. They are not the type of exercises I expect from a high-ranking university and are too simplistic.

Almost all of our coursework exercises have been in a form similar to a paper or a report. UCL engineering has an obsession with being concise, and do not seem to appreciate the benefits of longer forms of writing that go into more detail. I would appreciate it if a few of our coursework exercises were in the form of technical essays where you have more room to explore ideas in more detail and motivate ideas. Not to say that there are not benefits to being concise, but currently it feels like there is too much pressure to cram as much information into each sentence as possible. There is too much focus on stating what is the case without explaining why something is the case, and there is not space to freely explore a topic.

For an example of what I mean by an exploration into a topic, consider the video essay on splines by Freya Holm\'{e}r~\cite{Splines}. If the only goal was to just deliver content about the different types of continuity in splines, the end result would be very different and much shorter. This video has a narrative, it is framed as a journey of discovery into the topic, and this is what is missing when too much focus is put on being concise. When I write about technical topics recreationally, I create something much closer in style to this video on splines. I think there is utility in including such forms of writing as part of the course, especially for the longer individual project. We should at least be given the choice to do this if we want to.

Many of my friends and I have had issues with page limits. I cannot remember where from, but many of my friend group were under the impression that our individual project would have a 50 page limit. We thought that might be a bit tight but we could make it work, and when we found out the limit was actually 20 pages including appendices, we were shocked. We complained to the departmental tutor about the 20 page limit on the individual project, and they were insistent that this was not a problem, and that writing concisely was important. They said that sometimes academics could spend years on a paper and fit everything they covered into ten pages\footnote{I am highly skeptical an academic could produce one paper in several years without getting fired for lack of output, this seems more like an attempt to dismiss our problems.}. I understand the desire for us to write concisely, but this seems needlessly restrictive and makes it much harder to demonstrate the effort we have put in.

For example in our new and renewable energy systems coursework, I found the optimal tilt angles for a solar panel where the tilt angle was changed twice per year, and when they should be changed. I explored 10,000 configurations with Excel to cover the three dimensional configuration space, something that no one else bothered to do. The only mention of this analysis ended up as in a single line footnote due to the very tight requirements on space enforced on us. Despite having a similar level of brevity throughout the document, the marker still complained multiple times about how I was not concise enough.

My other main complaint about the coursework exercises is that they are extremely closed. My friends and I would try to find any opportunity to take things further and go into depth, but this is not encouraged. Firstly, there simply is not enough space to include content that strays any distance from the brief due to the very tight page and word limits. Secondly, we get no additional marks for this, there is no incentive for being creative or having original ideas. With the calibre of students that UCL attracts, the department should expect to get students who want to fully engage in the material and they should support this. The coursework exercises feel claustrophobic and stifling.

The finite element methods coursework was a good example of this. I was hoping to learn about hp-adaptive schemes or dynamic meshes, and at the very least use higher order elements. Instead we worked with the second simplest element (a first order quadrilateral) and used only three elements. We studied an extremely simple bracket and performed the most basic analysis into the deformation and strain on it. The scope was very narrow and the brief allowed for very little deviation. Most of the coursework exercises were similar with no room for imagination, and each submission would be very similar. This is a very boring way to do coursework.

If I were the author of the finite element method coursework, I would have made it much closer to a real-life situation. The students would be given a picture of a bracket and be told to use finite element modelling to suggest improvements. Here the students would need to decide for themselves what analysis they wanted to do, how they wanted to model it, and what direction they wanted to take it in. It is extremely open and allows students to be more ambitious in their approach. I would have thought the markers would be really bored reading essentially the same submission over and over, I do not understand why they would not want to have a wide variety of answers.

I understand that the markers are under pressure to mark quickly, but it has fallen below satisfactory. While frustrating for the department and the markers, they need to accept that marking takes as long as it takes and more time needs to be dedicated. The department has a rule of getting results back within some fixed time period, but it would be better if they relax that policy and mark our work properly. Pushing results back to the end of the year when exam results are released would even be reasonable in my opinion if this time is necessary.

\subsection{Coursework Weightings and Timings}

The weightings of the coursework exercises do not feel like they have been thought through very well. The amount of time we are supposed to spend on each coursework is unrealistic and some coursework is barely worth doing. Here I expand on how the engineering department gets the balance wrong, how they are confused about coursework, and how to fix it.

Some coursework exercises are simply not worth caring about. Each module is worth 15 credits out of a total of 180 which is 8.33\% which is already quite small. As part of the thermodynamics and turbomachinery module we have a literature review which is worth 8.7\% of the module, which means the whole thing is worth 0.725\%. Given that a low effort will get around 60\% and a huge effort will give around 80\%, this means the affect on the total grade will be 0.145\%. We have other stuff to do, many of my friends and I just took the 0.145\% hit and did terribly, why would we do anything else? This problem is made worse by the points I made in section~\ref{sec:MarkingOfCoursework} where the range of possible marks is very narrow, and could be alleviated by using the full range from 0-100\%. This coursework deadline was also very close to another coursework deadline, this pushed it even further down the list of priorities.

The thermodynamics and turbomachinery weightings were strange beyond this one coursework exercise. We have a presentation based on the literature review which is worth 8.8\%, and a coursework about fuel cells worth 17.5\%. The fuel cells coursework was due on the Monday after a Friday deadline for the group design project. Everyone I know did this in a hurry over the weekend\footnote{The group design project was worth far more than the fuel cells coursework.}, and it does not seem like there was any plan on when we would do this. Given that there was only one lecture on fuel cells, it was unrelated to the rest of the module, and fuel cells were covered in the new and renewable energy systems module, what was the point of this coursework? They should have just made the literature review and presentation worth more and got rid of the fuel cells coursework. I'm not sure how this made it past a peer review process or if there is a peer review process at all.

The lecturers do not understand the perspective of the students when it comes to coursework. In a conversation with one of the lecturers, they said they did not anticipate us caring as much as we did about the task, it was just meant to be a quick little thing. At any given point, there is only one thing that we can control that affects our grade, and that is the current coursework exercise that we have been assigned. Of course we are going to dedicate all of our time and effort towards it, of course we are going to agonize over every little detail.

The result of this way of thinking is that lecturers and the department have a strange idea about deadlines. One lecturer casually mentioned how they were going to release the coursework two weeks before the deadline, and they were shocked when we all immediately pushed back against that. This coursework was worth 25\% of the module, they should not have been surprised that we wanted more than two weeks. This has been an issue with other coursework exercises, and one time a lecturer even got into some trouble with the administration when they released a coursework too early (we had asked them to do this as the original timeline was absurd).

The coursework weightings do not line up well with how long they take either. Apparently one unit is meant to correspond to 10 hours of work, and after subtracting lecture time that means a 50\% coursework should take around 50 hours to complete. This means that something such as the finite element coursework should have been finished in well under a week, although if you were to try this then your submission would be terrible. It turns out we are not meant to care that much about coursework, and I fully wasted my time spending over 200 hours actually submitting something of high quality\footnote{I only got 76\% on the coursework, although this is 4\% higher than anyone else I know.}. It was not just me spending this much time, almost all of my friend group spent several times longer on each coursework than we were meant to as well.

Based on the data from my friend group, the marks gained per unit time drops off to almost 0 past a certain point. In conversations with the departmental tutor it was revealed that the length of a coursework only has a very small correlation with how long it takes, its weighting, and the page limit. A sentiment I have got from several staff members is that we should not expect marks to correspond to effort which seems strange to me. While it is true that effort can be spent on something terrible, I still maintain that this is a disconnect from the experience of the student and is a problem.

I would like to see the coursework exercises weighted much higher and less of an emphasis placed on assessment through examinations. The number of hours of study would not change significantly as the exams would still require preparing for the same amount of content, the stakes would just be lower. The weighting of the coursework exercises is too low to match the time investment, and this would bring it more in line. Coursework also have a much higher ceiling than examinations\footnote{This is true if the coursework exercises have been designed well, something that is not currently true at UCL.}, and much more advanced concepts can be covered. Personally I would make the contribution of coursework to examinations in a 4:1 ratio.

\subsection{Marking Feedback and Lecturer Interactions}

The feedback we get with our coursework results is often wrong, insufficient, and does not help us improve. Some lecturers are particularly frustrating to deal with when trying to get clarification, and unhelpful when we follow up on the marks and feedback we have received.

When we complain about issues with the feedback to our coursework submissions we very rarely get our marks changed. This is the case even when we are objectively correct about issues with the marking. By accepting that significant errors were made in our work, they are implicitly accepting that the marking process allows such significant errors. I believe that they just do not want to remark everything, and therefore the decision to reject the evidence of poor marking is a foregone conclusion.

In my finite element method coursework I lost marks for using inline maths such as ``The Pythagorean theorem is given by $a^2 + b^2 = c^2$" instead of ``The Pythagorean theorem is given by equation (\#)", and including an equation outside the body of the text. This is incredibly standard in technical writing and there was no mention about this in the brief. If lectures are going to pull moves like this then they need to give a comprehensive style guide as otherwise they are asking us to read their mind. I did not even attempt to challenge the marking for this coursework despite my strong case as I knew I would get nowhere.

I ran into numerous nonsense problems such as this where the only way around it would be to send hundreds of emails to the lecturers asking about every detail. I knew the finite element method lecturer would give me problems so I did ask about details, and this resulted in him telling me he was not answering any more questions after only nine emails. The lecturer telling me something is in the brief when it demonstrably is not is frustrating and not helpful. Either they have terrible reading comprehension skills, a poor grasp on reality, or more likely they are just lazy. I would not be so annoyed about this if the lecturer did not care about these details, but the stupid thing is that they did.

Even when the feedback we receive is not incorrect, it is often not useful. Very often my friends and I have been given feedback such as ``very good" or ``excellent", yet have still lost marks. If an answer did not get full marks then there must be some flaw in it, the marker's feedback needs to state what that flat is, and justify the mark given. Feedback that translates the numerical score into a very brief description does not help us determine what went wrong or how to improve. It also puts us at a disadvantage when attempting to dispute our marks as we do not know the logic behind the mark and we cannot judge whether it is reasonable for ourselves. Please give us more detailed feedback.

The poster coursework for the materials and fatigue module had a review process that I want to discuss. We had the opportunity to send a single draft of our posters to the module coordinator, Chu Lun Alex Cheung, and also the teaching assistant for a review. This was very useful, and UCL should try to do more like this in the future. While this was greatly appreciated, there were still issues with this however. I asked a question about whether a section was suitable and got an answer that heavily suggested getting rid of it, although in my final feedback apparently I should have included such a section. This is an example of the random number generator marking at UCL where you are at the mercy of the whims of the marker. When I emailed about this along with several other issues I did not get a satisfactory response.