\section{Conclusions and Recommendations}
\label{sec:Conclusion}
% Lack of peer review
% Addition of lecture notes
% Range of examples of submissions
% Longer feedback for marking
% More time spent on marking
% All lecturers should watch a Paul Helier lecture and an Alex lecture

This section draws on the evidence presented in the previous sections, summarises recommendations for future policies, and also gives additional suggestions. The main ideas I discuss here are the implementation of review processes, a larger focus on teaching over research, and an overhaul in the approach to the specification and resources. Some of these proposed changes involve significant time and money investment, but they are made on the assumption that the department is serious about improving the quality of the course, and wants to justify UCL's position as a top university.

There are several processes that the lecturers cannot be trusted to do on their own. These include the production of resources, writing and construction of coursework briefs, marking of work, and writing examination papers. Reviewing processes should be put into place for all these documents and tasks, and this should be done to minimise conflicts of interest. For example, a lecturer is less likely tell another lecturer that they should make their own figures if they are going to be held to the same standard as their reviewee later on. The reviewers should not hold an attitude of ``good enough", but aim to find as many flaws as possible within reason\footnote{For example, when I am proof-reading the work of my peers, I point out when a non-breaking space should be used, such as 3 cm, but not when they have split an infinitive.}. There are some review processes in place already, for example the marking needs to be signed off by someone else. Given the huge number of errors that pass the review, the standards of review need to be increased significantly.

The resources in the department need such significant changes that it is almost worth starting again from scratch. Firstly, lecture notes should be produced, typeset in {\LaTeX} to a near publishable quality. These notes should be the main source of information for each module, and the lecture slides should be redesigned as presentation aids. Lecturers will almost certainly benefit from training on the design of presentation slides, and will likely need training in the use of {\LaTeX} and typesetting\footnote{I want to point out that in my undergraduate course we received very little training in either of these, yet everyone was still more than capable of figuring it out on their own. I do not trust the same thing to be the case with UCL engineering lecturers.}. Construction of these resources is not a quick job, and requires significant dedication\footnote{If the department want to be lazy, they can ask the students to create a section of the lecture notes as part of a coursework exercise. I guarantee this will result in higher quality notes than if the lecturers had to do it themselves.}. Most figures should be in vector formats and will need to be custom made, for example in Python, MATLAB, or \textsf{R} for plots, and Inkscape for other figures. An example all lecturers should follow is found in~\cite{OxfordNotes}. Lectures should also watch a Paul Hellier lecture to see how lecturing should be done.

The specification and focus for all modules needs to be changed significantly as currently the material is very surface level and basic. The calculations performed feel like the back of the envelope calculations done in the first few minutes of the design stage in a project, and should be made much more complicated. The scope of each module is very broad, and while that has its benefits, some breadth should be removed in favour of going into detail on a selection of topics. The difficulty is also a problem, and needs to be increased considerably. More focus should be put on the development of sophisticated mathematical models, and the advanced ideas at the cutting edge of the industry. This is the level of challenge I would expect at a university asking for BCC at A-level, not A*AA, make the content harder.

Coursework at UCL needs overhauling, in terms of the nature of the exercises, weighting, and marking. Currently it feels like busywork that could be done by any somewhat numerically competent person, and ChatGPT is able to answer the questions. Make the coursework exercises more open to allow us to be creative what direction we take them, and also to take the material as far as we want to. The marking of coursework needs an intensive review, and more time needs to be dedicated to it. The way marking is done leaves students playing a game with unreliable rules is unfair and needs to be remedied. Markers should stop skimming, make comments on the documents, and give much more detailed feedback. More time needs to be allocated to the marking of work, and more effort spent in review, perhaps even double marking all work. The department will not find a quick fix here, the problems are systemic and significant.

Coursework weightings need to be increased substantially as the current weightings do not represent the time effort needed. The importance of examinations should be reviewed given their numerous issues, for example some people are highly competent but do not perform well on examinations. I recommend a 4:1 ratio for the weighting of coursework and examinations respectively. For example I spent over 200 hours on a coursework that has half the weight of the individual project presentation. While presenting is important, I think the department does not understand the time investment necessary to produce a high quality coursework submission. An attitude shift needs to occur, as it appears we are not meant to care much at all about our coursework. {\LaTeX} should also be a requirement for at least one coursework as knowing the tools required to typeset a technical document properly is important. Learning more software would also improve the utility of the coursework exercises.

\newpage

All examination resources should be centralised and conveniently collated. UCL's current system using the library does not work and the resources uploaded to the Moodle are a mess. The past paper resources should also be the up to date version of the paper and mark scheme, and not the first drafts. Examinations should also follow the same format between years, and preferably standardisation between modules would also be nice. Being given a choice of questions would also be a change welcomed by students. The weighting of questions should be reviewed to ensure the same level of effort is required per mark. Question writing needs to be improved, and training for this will almost certainly be needed.

The department should do much more to listen to feedback\footnote{Merely collecting feedback does not count, it must be acted on and the response communicated}. Their current attempts are pathetic and the administration needs to do a lot more to keep the lecturers in line. When students collectively come to the department with a problem, the department should not simply dismiss it. A lot of work needs to be done if the department want students to trust them and the systems in place. For example, I know that new staff are split into research focused and teaching focused roles, but the main problems are with the current staff. It is clear that many of the current staff do not care about teaching, and I think they should be left to do their research if they want. More needs to be done to incentivise teaching roles, including training and extra pay.

The academic integrity policy needs to be rethought. The priority should be on malicious behaviour that would lead to students getting higher marks than they deserve. Checking results, asking for help from peers, and using tools such as ChatGPT should be recognised as assistive, but the students still have sole authorship over their work. The lack of being able to demonstrate any individuality while remaining within the scope of the coursework briefs also allows for collusion as everyone has to do the same thing anyway. The lack of care in marking means that dodgy references and lazy mathematics are almost certainly going to be missed. The department should either admit that they are too lazy to bother checking, or stop the facade that they care. In online examinations, collusion can be significantly reduced by curving the results\footnote{Firstly, one person doing better via cheating means everyone else does worse, but without curving everyone is completely independent and there is no effect on anyone else. Secondly, there is a selfish reason, as if a person helps another, that person does worse.}.

In summary, the department has many issues to fix. I think most of their problems come from a lack of caring, and not holding themselves to high enough standards. The department needs to recognise the problems, listen to feedback, and be prepared to take significant action in order to fix them.